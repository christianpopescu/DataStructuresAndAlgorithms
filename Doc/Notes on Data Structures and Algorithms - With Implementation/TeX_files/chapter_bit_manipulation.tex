% !TeX spellcheck = en_US
\chapter{Bit Manipulation}
\lstset { %
	language=C++,
	backgroundcolor=\color{black!5}, % set backgroundcolor
	basicstyle=\footnotesize,% basic font setting
	tabsize=4,
}

[\cite[See][- 2.3 Bit Manipulations]{LaaksonenGuideToCompetitiveProgramming}]

\section{Some interesting functions and examples}

\begin{itemize}
	\item set the k\textsuperscript{th} bit of x to 1. 
	\begin{lstlisting}
	\x|(1<<k)
	\end{lstlisting}
	\item set the k\textsuperscript{th} bit of x to 0.
	\begin{lstlisting}
	x&~(1<<k)
	\end{lstlisting}
	\item inverts the k\textsuperscript{th} bit of x to 0.
	\begin{lstlisting}
	x^(1<<k)
	\end{lstlisting}
	
	Still to do what is on page 22
		
\end{itemize}




The following bloc prints the bit representation of an integer:
\lstset { %
	language=C++,
	backgroundcolor=\color{black!5}, % set backgroundcolor
	basicstyle=\footnotesize,% basic font setting
	tabsize=4,
}

\color{blue}
\begin{lstlisting}[frame=single]
for (int k = 31; k >= 0; k--) {
	if (x&(1<<k)) cout << "1"
	else cout << "0"		
}

\end{lstlisting}

\color{black}
\section{Quick sort}
